%%%%%%%%%%%%%%%%%%%%%%%%%%%%%%%%%%%%%%%%%
% Long Professional Curriculum Vitae
% LaTeX Template
% Version 1.1 (9/12/12)
%
% This template has been downloaded from:
% http://www.latextemplates.com
%
% Original author:
% Rensselaer Polytechnic Institute (http://www.rpi.edu/dept/arc/training/latex/resumes/)
%
% Important note:
% This template requires the res.cls file to be in the same directory as the
% .tex file. The res.cls file provides the resume style used for structuring the
% document.
%
%%%%%%%%%%%%%%%%%%%%%%%%%%%%%%%%%%%%%%%%%

%----------------------------------------------------------------------------------------
%	PACKAGES AND OTHER DOCUMENT CONFIGURATIONS
%----------------------------------------------------------------------------------------

\documentclass[10pt]{res} % Use the res.cls style, the font size can be changed to 11pt or 12pt here
\usepackage[utf8]{inputenc} %codification, acentos y demas...

\usepackage{helvet} % Default font is the helvetica postscript font
%\usepackage{newcent} % To change the default font to the new century schoolbook postscript font uncomment this line and comment the one above

\newsectionwidth{0pt} % Stops section indenting

\begin{document}

%----------------------------------------------------------------------------------------
%	YOUR NAME AND ADDRESS(ES) SECTION
%----------------------------------------------------------------------------------------

\name{Paracuellos Gutiérrez, Sergio\\ \\} % Your name at the top

% If you don't want one of the addresses, simply remove all the text in the first or second \address{} bracket

\address{ Concejo de Elcano 12, 1º-1\\ 31016, Pamplona\\ (+34) 655545852 \\ sergio.paracuellos@gmail.com \\ https://github.com/paraka \\ https://www.linkedin.com/in/sergio-paracuellos-23a34a17} % Your address 2

%----------------------------------------------------------------------------------------

\begin{resume}

%----------------------------------------------------------------------------------------
%	OBJECTIVE SECTION
%----------------------------------------------------------------------------------------

%\section{\centerline{OBJETIVO}}

%\vspace{8pt} % Gap between title and text

%Formar parte de la plantilla del Centro de Respuesta a incidentes en Tecnológias de la información (INTECO-CERT) con el fin de
%aprender y desarrollar mis capacidades en el ámbito de la seguridad informática.\\ 

%----------------------------------------------------------------------------------------
%	EDUCATION SECTION
%----------------------------------------------------------------------------------------

\section{\centerline{Studies}} 

\vspace{8pt} % Gap between title and text

{\sl PikeOS Driver Training} \\
San Fernando de Henares, Madrid \hfill September 18th-20th 2018

{\sl Tailored VxWorks 653 3.0.1.1 MCE Essentials and Advanced Topics} \\
San Fernando de Henares, Madrid \hfill May 29th-31th 2018

{\sl Japanese Graduate}, 
Kyoto Institute of Culture and Language \\ 
24 Kamihate-cho, Kitashirakawa, Sakyo-ku, Kyoto, 606-8252 JAPAN \hfill March 2009 - July 2010 
 
{\sl IT Management}, MBA-TECH\\ 
Universidad de San Jorge Parque Tecnológico Walqa, 22197 Huesca \hfill January 2008 - June 2008

{\sl ITIL v3 Certified} \\ 
Exin: Examination Institute for Information Science \hfill March 2008

{\sl Master Degree in Computer Engineering. Systems engineering speciality} \\
Universidad de Zaragoza, Zaragoza \hfill September 2003

{\sl J2EE distributed application and Webservices} \\
Engineering School, La Almunia \hfill December 2002 (Duration 56 hours)

{\sl Autocad v.12 course} \\
Colegio Sagrado Corazón de Jesús, Zaragoza \hfill June 1996 (Duration 130 hours)

{\sl Programming and Telecomunication introduction course} \\
CLD División Informática, Zaragoza \hfill May 1994 (Duration 60 hours)

{\sl Operating Systems and DBASE IV Programming course} \\
CLD División Informática, Zaragoza \hfill May 1993 (Duration 60 hours)


%----------------------------------------------------------------------------------------
 
\vspace{0.2in} % Some whitespace between sections

%----------------------------------------------------------------------------------------
%	PROFESSIONAL EXPERIENCE SECTION
%----------------------------------------------------------------------------------------

\section{\centerline{WORK EXPERIENCE}} 

\vspace{8pt} % Gap between title and text

{\sl Aingura IIoT (ETXE-TAR group)} \hfill November 2018 -- Current\\
\hfill Senior Embedded software engineer
\begin{itemize} \itemsep -2pt % Reduce space between items
\item BSP's development using GNU/Linux: yocto.
\item Bare metal driver development for adc's (energy and vibration) over arm R5 making use of libmetal.
\item Linux remoteproc API for controlling arm R5 cores of MPSoC from linux.
\item General Linux driver development (i2c, spi, iio, dma...).
\item FPGA design modifications using Xilinx Vivado for MPSoC ultrascale.
\item Client applications development: C, C++ (c++17 standard, gcc 8.3.0).
\item Deploy of CI/CD ecosystem for the target artifacts build system and tests on host and on target (docker + gitlab + jenkins).
\item Main hardware platforms: iveia Atlas II Z8 Zynq ultrascale+MPSoC SoM.
\item All developments have as main programming languages C, C++, python and bash.
\end{itemize}

{\sl Orbital Critical Systems} \hfill January 2017 -- November 2018\\
\hfill Senior Embedded software engineer
\begin{itemize} \itemsep -2pt % Reduce space between items
\item Development of Real Time Operating System ARINC-653 compatible for targets using SOC Zynq 7000.
\item Development of drivers from scratch for the operating system: i2c, spi, uart, dma, fpga ip cores, ttethernet.
\item Development of a minimal C library for the system.
\item Development of SDK and build system using cmake in several projects.
\item Development of internal unit test library taking into account DO-178C certification.
\item BSP's development and modification for different RTOS: RTEMS, PikeOS, VxWorks 653.
\item Development of applications to run over different RTOS: RTEMS, PikeOS, VxWorks 653.
\item Code coverage analysis to cover avionics DO-178C certifications: internal library for different scopes using gcov.
\item Linux kernel modules development for Xilinx FPGA ip cores.
\item Simple FPGA designs using Xilinx Vivado for SoC Zynq7000 (zc7045, zedboard, microzed, minized).
\item Debugging tools: gdb, xilinx debug tools (xsdb), lauterbach.
\item Main hardware platforms: Zynq7000, QorlQ T2080.
\item All developments have as main programming languages C, C++, python and bash.
\end{itemize}

{\sl Cemitec} \hfill January 2016 -- January 2017\\
\hfill Embedded software engineer 
\begin{itemize} \itemsep -2pt % Reduce space between items
\item BSP's development using GNU/Linux: yocto, debian, ubuntu etc: MX6, Beaglebone black...
\item Linux kernel modules develpment: creation and modification.
\item Linux on FPGAs: Xilinx Zynq.
\item Firmware applications development for clients: protocol + GUI: C, C++, Qt5.
\item IoT: Border router over linux: 6LowPAN 6lbr + BBB. COAP: libcoap, MQTT: broker mosquitto, libmosquitopp.
\end{itemize}

{\sl Scati Labs S.L.} \hfill September 2013 -- December 2015\\
\hfill QA and Developer Engineer
\begin{itemize} \itemsep -2pt % Reduce space between items
\item Ensuring product quality at various levels: source code, functionality, scalability ...
\item C ++ / python code reviews (among others) using Reviewboard tool.
\item Monitoring Linux / Windows systems (PRTG / Icinga). Development of multiple scripts to perform these tasks (bash, perl, python).
\item Performance and functionality tests automation in Windows/Linux environments.
\item Nontrivial unit test development using gtest and gmock C++ tools.
\item Development of functionalities of the company products according to customer needs (C++ and python).
\item Development of compile solutions for C++ projects using cmake in both Windows and Linux enviroments.
\item Development of solutions with WebServices using C ++ gsoap libraries (onvif integration, own web services).
\item Migration of the repositories of the company (more over 11 years) from cvs to git.
\end{itemize}

{\sl Zitralia Seguridad Informática S.L.} \hfill Febrary 2010 -- May 2013\\
\hfill Developer Engineer
\begin{itemize} \itemsep -2pt % Reduce space between items
\item Development of a rootkit with aim of getting all the possible information of a target computer with Linux operating system, and transmit it in real time to some other place where could be analyzed.
\item Development of a Ubuntu GNU/Linux based system from scratch (debootstrap). 
\item Development of programs for that system using perl, C++, C, bash and GTK+.
\item Development of tools to burn that system into pendrives for linux (Perl and GTK+ )and windows (MFC and COLINUX) enviroment using LUKS encryption techniques.
\item Linux XFCE 4.10 Desktop adaptation for that system.
\item Adaptations of the system to be started as a program in a windows environment using virtualbox in headless mode. Validation own application using MFC Windows widgets.
\item Web development tool using php, webmin and mysql databases for configuration and generation of the "on demand" operating system.
\item Adaptation of the system to be used in the cloud using HTML5 guacamole over Ulteo (RDP).
\item UNIX/Linux systems administration to support various projects (automation of tasks, backups ...).
\item Mobile application development related to the pharmaceutical industry for both Android and iPhone/iPad smartphones.
\end{itemize}

{\sl Supermarket worker in Kyoto (Japan)} \hfill December 2009 -- October 2010\\
\begin{itemize} \itemsep -2pt % Reduce space between items
\item Meat, fish and vegetables manipulation.
\item Typical store and warehouse tasks.
\item Truck loading and distribution.
\end{itemize} 

{\sl Zitralia Seguridad Informatica S.L} \hfill July 2005 -- March 2009\\
\hfill Developer Engineer
\begin{itemize} \itemsep -2pt % Reduce space between items
\item Development of an encrypted file system for the linux kernel using Fist.
\item Development of several kernel modules for different purposes: authentication, network TCP frames modification, different uses of linux kernel cryptoAPI, several functions of linux kernel file and memory systems. 
\item Bash and perl scripting for different automation stuff.
\item ClI and GUI (GTK+) tools written in Perl.
\item GNU/Linux servers administration to support different developing projects (subversion, firewall, apache, dhcp, postfix...)
\end{itemize}

{\sl Escuela Universitaria Politécnica de la Almunia de Doña Godina} \hfill September 2002 -- July 2003\\
\hfill Developer 
\begin{itemize} \itemsep -2pt % Reduce space between items
\item Development a Java to .NET framework compiler written in java language using Java SableCC 3.0. It was developed from grammar to the IL intermediate code generation for the .NET platform and worked on both Linux (Mono project) and Windows (.NET).
\item This development was my final university project and obtained a rating of outstanding.
\end{itemize}

%----------------------------------------------------------------------------------------

\vspace{0.2in} % Some whitespace between sections

%----------------------------------------------------------------------------------------
%	LENGUAJES DE PROGRAMACION
%----------------------------------------------------------------------------------------

\section{\centerline{PROGRAMMING SKILLS}}

\vspace{15pt} % Gap between title and text
\begin{itemize} \itemsep -2pt % Reduce space between items
\item C++, C, Perl, Java, Ada, Python, C\# and Component Pascal.
\item Grafic Libraries: Qt5, GTK+ (C, C\#, Perl and Python), Windows MFC.
\item Scripts programming: bash, perl, python.
\item Web programming: PHP (PHP Hypertext Preprocessor), MySQL, HTML and CSS (Cascading Style Sheets).
\item Smartphone Applications development: android, XCODE 4 (iOS).
\end{itemize}

%----------------------------------------------------------------------------------------

\vspace{0.2in} % Some whitespace between sections

%----------------------------------------------------------------------------------------
%	SSOO
%----------------------------------------------------------------------------------------

\section{\centerline{OPERATING SYSTEMS}} 

\vspace{15pt} % Gap between title and text
\begin{itemize} \itemsep -2pt % Reduce space between items
\item Linux kernel: eudyptula challenge completed (number 137).
\item Debian GNU/Linux (and derivatives Ubuntu...): Advanced user and administration.
\item Fedora Core from version 3 until present versions / CentOS: Advanced user and administration.
\item FreeBSD.
\item RTOS: RTEMS, PikeOS, VxWorks 653.
\item Windows 98/2000/XP/7: Advanced user and administration.
\item MacOS X: Advanced user and administration.
\end{itemize}

%----------------------------------------------------------------------------------------

\vspace{0.2in} % Some whitespace between sections

%----------------------------------------------------------------------------------------
%	TRabajo en GRUPO
%----------------------------------------------------------------------------------------

\section{\centerline{GROUP WORK TOOLS}} 
\vspace{15pt} % Gap between title and text
\begin{itemize} \itemsep -2pt % Reduce space between items
\item GIT, subversion (SVN), CVS: version control systems.
\item Bugzilla, Mantis, Redmine, Jira: Task and Project administration.
\item Wiki, Confluence: Web coordination.
\end{itemize}

%----------------------------------------------------------------------------------------

\vspace{0.2in} % Some whitespace between sections

\section{\centerline{DOCUMENTATION}} 
\vspace{15pt} % Gap between title and text
\begin{itemize} \itemsep -2pt % Reduce space between items
\item Document creation with XML, SGML and LATEX.
\item Experience with Docbook DTD.
\end{itemize}


%----------------------------------------------------------------------------------------

\vspace{0.2in} % Some whitespace between sections

\section{\centerline{NETWORK ADMINISTRATION}} 
\vspace{15pt} % Gap between title and text
\begin{itemize} \itemsep -2pt % Reduce space between items
\item Apache Server administration: Basic setup, virtual domains, security…
\item Net Services: OpenSSH, FTP, DHCP...
\item Administration of own Server with 25 users since 2002. Mail (POP3, SMTP, mailman), web Server (apache), firewall (iptables), cvs, user accounts, security backups…
\end{itemize}

%----------------------------------------------------------------------------------------
%	IDIOMAS SECTION
%----------------------------------------------------------------------------------------

\section{\centerline{LANGUAGES}} 
\vspace{15pt} % Gap between title and text

\begin{itemize}
\item Avearage level in spoken and written english. Several month stays in English and Irish family environments. Average level certificate: Carraig Linguistic Services (Cork, ireland).  
\item Average level in spoken and written japanese. Two years living in Kyoto. Japanese-Language Profilence Test: N2 (JLPT 2). 
\end{itemize}

%----------------------------------------------------------------------------------------

\vspace{0.2in} % Some whitespace between sections

\section{\centerline{OTHER}} 
\vspace{15pt} % Gap between title and text

\begin{itemize}
\item B driving license and own car.
\end{itemize}

%----------------------------------------------------------------------------------------
%	INTERESTS SECTION
%----------------------------------------------------------------------------------------

\section{\centerline{OTHER INTEREST ACTIVITIES}} 
\vspace{15pt} % Reduce space between section title and contents
\begin{itemize}
\item Flamenco guitar lover and player since 2004.
\item Active member of La Almunia Linux User Group (GrULLA), (2001-2007).
\item Member of User Group Hispalinux since 2002. 
\item Took part in the organization of the V Hispalinux Congress at UAM (Universidad Autonoma de Madrid)
\end{itemize} 

%----------------------------------------------------------------------------------------

\end{resume} 
\end{document}
